\documentclass[../Main.tex]{subfiles}

\begin{document}
\begin{refsection}

\chapter{Returns to Education}
    


\intro{ 
\begin{itemize}
    \item \textit{Does education create an increase in wages?}
    \item \textit{Can education decrease inequality?}
\end{itemize}
}

\section{Economic Theory}

There are two main theories underpinning potential effects of education on the wage equation, each with their own nuance: 

\thm{Human capital}{The Human capital theory, the older more established theory,
suggests that education is useful \textit{in of itself}, since school provides
skills such as writing, reading, typing, arithmetic, etc. Skills which are
transferable and useful in the workplace.}


\thm{Signalling}{The signalling theory is a result of more recent developments
in game theory and behavioural economics, it suggests the skills from education
are not always useful in of themselves, instead training and tenure being more
useful, especially for firms hiring in the more specialised roles of a modern
quaternary economy. Education is assumed to by a \textbf{type signal} whereby
getting qualifications, budding employees signal to employers that they are of a
different \textit{type}\footnote{This is game theory jargon but just consider a
unique characteristic of a worker} to others by being more skilled than their
counterparts.}

\newpage
\section{Empirical Framework}

Economists tend to utilise a regression equation to estimate the impact of education on the wages of a given individual: 
\begin{equation}
    Y = \alpha + \beta S + \delta X + \epsilon
\end{equation}

\begin{itemize}
    \item $X$ is often used as a catchall vector for the control variables such as age,
    gender, ethnicity etc. 
    \item $S$ here is the \textit{regressor}\footnote{Regressor simply refers to
    a variable within the regression equation. Inversely, Regressand refers to
    the variable being predicted by the regression equation.}, in this case, it is years of schooling.
    \item $Y$ is the \textit{regressand}, the hourly wage.
    \item $\alpha$ is the intercept. Intuitively, it is the wage estimated by
    the lowest education for the control individual, where all dummies in $X$
    are set to 0.\footnote{For example, say $X$ consisted of two dummies
    $\text{female}$ and $\text{non-white}$, the control individual where all
    dummies are null would be a white male with no education.} \item $\beta$ is
    the effect of changing values of $S$ on $Y$, and is the
    predicted impact of a change in schooling years on the wage equation.
\end{itemize}

\subsection{Testing Statistical Significance}
    
The \textbf{standard error} is a commonly used measure of statistical precisions
with which coefficients are estimated. It is important to statistically test the
coefficients to ensure that they are meaningful and robust.

The t-statistic is a measure of the relative size of the coefficient to the
standard error: 
\begin{equation}
    t = \frac{\hat{\beta} - \beta_0}{\text{se}(\hat{\beta})}
\end{equation}
With a null hypothesis that $\beta = 0$; the t-statistics simply reduces to $t =
\hat{\beta} \div \text{se}(\hat{\beta})$. We can use a look up table to
determine whether the p-value; 0.05 is the generally accepted value for
statistical significance.\footnote{A nice heuristic: if the absolute value of
the t-statistic is greater than 1.96, than a null hypothesis of $\beta_0$ can be
rejected.}

However, researchers be weary, inferring causality from estimates of
coefficients is still problematic, even if the t-statistic determines them to be
statistically significant. The main issue is that of \textbf{endogenity},
identified where the regressor correlates strongly with the error term,
indicating multiple potential issues with the regression or the data: 
\begin{description}
    \item[Omitted Variable Bias] Where a missing variable correlates with both the regressor$S$ and the regressand $Y$. 
    \item[Reverse Causality] Where the implied causality is between the
    regressor and regressand where in fact its the other way round.
    \item[Simultaneity] Where both the regressor and regressand dynamically
    interact with each other. \end{description}

Any sources of endogeneity violate the
\href{https://en.wikipedia.org/wiki/Gauss%E2%80%93Markov_theorem#:~:text=the%20original%20equation.-,Strict%20exogeneity,-%5Bedit%5D}{Gauss-Markov Theorem} assumption of strict exogeneity with the error term, resulting in an
upward bias on $\beta$.

\vfill
    \rmkb{\textcolor{red}{\textbf{ALL REGRESSIONS FACE THE CAUSATION PROBLEM.
    SEPERATING CAUSATION FROM CORRELATION IS A THEORETICAL ENDEAVOUR.}}} \vfill
\newpage
\section{Empirical Approaches}

An ideal empirical methodology would involve creating exogenous variation in the
regressor $S$ that is uncorrelated with the regressand $Y$ after conditioning
with controls $X$. This would allow a researcher to effectively create
\textit{control} and \textit{treatments} groups. However usually creating such
perfect experiments in an economic setting raises ethical concerns; the obvious
being it is immoral to actively deny a child schooling to assess future outcomes
knowing that it would likely be to the detriment. To \textit{partially-}solve
this econometricians use sources of plausible exogeneity to create
\textit{quasi-natural} experiments.

\subsection{Instrumental Variable Approach (2SLS)}

This method of creating quasi-experimental conditions by identifying a plausible
source of exogeneity. By first predicting the regressor with an instrument, $Z$, and
then predicting the regressand using the predicted value of the regressor, the researcher can isolate the impact of the regressor on the regressand if \textbf{two assumptions} are satisfied.

\asumr{Relevance}{relevance}{The instrument must be \textbf{highly} correlated with the
regressor. This assumption can be validated by checking that the instruments
coefficient on the regressand is statistically significant ($\alpha_1$).}

\asumr{Exclusion Restriction}{exclusion}{The instrument must be uncorrelated with the
regressand except through its impact on the endogenous regressor.
\textcolor{red}{\textbf{This is UNTESTABLE!}} }

\paragraph{Stage One: Predicting the regressor with the Instrument} 
\begin{equation}
    \hat{S} = \alpha_0 + \alpha_1 Z + \alpha_2 X + \mu
\label{eq:2sls stage one}
\end{equation}
\paragraph{Stage Two: Predicting Regressand using Stage One}
\begin{equation}
    Y = \beta_0 + \beta_1 \hat{S} + \beta_2 X + \epsilon
\end{equation}


\exm{\textcite{angrist_does_1991}}{
Used an instrument variable approach, exploiting variation in institutional
rules which create quasi-random variation in schooling. Using men between 1930
and 1959, those born in an earlier quarter of the year have slightly less
schooling than those born later in the year due to compulsory school leaving age
policies set by the government: \textit{quarter of birth} therefore a plausible
instrument.

Their strategy requires that variation in schooling years, $S$, by quarter of
birth only results from institutional variation in policy. However there is
potential mechanisms that would cause quarter of birth to be correlated with
ability, for example, children older in the year may be of greater mental and
physical ability which could have tangible impacts on the wage equation and thus
potentially endogenous with wage outcomes and violation of Assumption
\ref{asum:exclusion}.}

\subsection{Sibling / Twin Studies}

Twin studies are widely used in econometrics to control for unobserved genetic
and environmental factors. Since twins share the same genes (identical twins) or
a significant portion of them (fraternal twins) and often grow up in the same
household, differences in outcomes (such as wages) can be attributed more
directly to differences in treatment (such as education). Consider a simple
regression for each twin:

\begin{equation}
    Y_i = \alpha_i + \beta S_i + \epsilon
\end{equation}

\asumr{Identical Error Terms}{twinepsilon}{In a twin study it is assumed that
the twin are identical in their error terms implying their otherwise
unobservable traits are identical.}

Thus the difference between outcomes between twins is found using:

\begin{equation}
    \begin{split}
        Y_1 - Y_0 &= (\alpha_1 - \alpha_0) + \beta(S_1 - S_0) + \epsilon - \epsilon \\ 
        Y_1 - Y_0 &= (\alpha_1 - \alpha_0) +\beta(S_1 - S_0)
    \end{split}
\end{equation}

\begin{itemize}

    \item $\alpha_i$ is the twin-specific fixed effects, this attempts to
    control for observable characteristics of each twin that may differ their
    wage outcomes outside of amount of schooling, 

    \item $\beta$ represents the policy effect of the difference in schooling years.

    \item $\epsilon$ represents the identical error term between twins. 
    
    
\end{itemize}

\exm{\textcite{card_chapter_1999}}{
After comparing the effectiveness of different empirical strategies, David Card
finds that twin studies yield similar results to
\href{https://en.wikipedia.org/wiki/Mincer_earnings_function}{Mincer}
regressions estimated by OLS. That is, siblings with greater education tend to
have greater wages on average. }

Despite the family background aspect of twin studies being an important role in
controlling for potential unobserved characteristics it is not a perfect
control. It may not wholly capture individual ability and the strong assumption
of identical error terms (\ref{asum:twinepsilon}) is still not likely in most
experiments. 

\section{Practical Questions} 

\textcite{angrist_does_1991} use an instrumental variables methodology to
estimate the effect of an additional year of schooling on earnings. The
instrumental variable they adopt combines the child’s month of birth with an
institutional rule that all children must stay at school until they are aged 16
(at the time of the study). In the UK, all children start school in the
September that they are aged 4. This means that 

\begin{itemize} 

\item children born in August start school when they have just turned 4 

\item children born in September start school when they are nearly five 

\end{itemize} 

In the study, all children had to stay at school until they are 16. This means
that children born in August have 1 additional year of compulsory schooling. The
instrumental variable is defined to take the value of 1 if the child is bornin
August and 0 if born in September.

\begin{enumerate}

    \item \begin{enumerate}
    
        \item Define the treatment and control groups in this study

        \begin{answer}
        
            The treatment group is those born in August vs those born
            earlier since the former allowed for an additional year of
            schooling

        \end{answer}
        
        \item What are the two assumptions inherent in the IV estimation

        \begin{answer}
            
            \textbf{Relevance} - the instrument (the random birth month)
            must be highly correlated to years of compulsory schooling, likely
            to be true to regulatory framework.
            
           \textbf{Exclusion Restriction} - the instrument must only
            effect the outcome variable through its impact on the regressor and
            has no indirect impact.

        \end{answer}
            
        \item How can you test whether these assumptions are correct? 

        \begin{answer}

            \begin{description}
            
                \item[Relevance] can be tested using a t-statistic test when
                regressing the instruments impact on compulsory schooling: if
                significantly
                non-zero we can more confidently assume the IV is relevant.
    
                \item[Exclusion Restriction] cannot be directly tested as
                indirect effects are not able to be isolated.

            \end{description}

        \end{answer}
            
        \item List two potential failures of the IV assumptions in this case

        \begin{answer}
        
        The most likely assumption to be violated is the exclusion
        restriction - if the birth month effects the earnings later in life
        through other unobserved factors. For example, if those born later in
        the year are persistently less confident than older pupils who were at
        later states of physical and mental development throughout education
        which could have an impact on long term wage outcomes. Alternatively,
        knowing that children born earlier are more likely to get more
        schooling, parents may strategically choose to try and have there
        children born around the beginning of the academic year as to get an
        advantage, the characteristics of such parents could effect the child's
        upbringing.

        \end{answer}

    \end{enumerate}

    \item Understanding variation we see in the data. Another way to think of
    the return to education is to understand the effect that parents’ education
    has on the outcomes of their children. 

    \begin{enumerate}
    
        \item Can you use sibling/twin studies methodology? Write out an
        equation to explain why.

        \begin{answer} 
            No, since twin/siblings have parents of the same education level.
        \end{answer}
        
        \item If we collected a dataset of you and your fellow Econ students,
        then in 30 years measured again, for those of you who had children, the
        achievement of the children in their tests at school. Would this be
        useful data to study the effect of parent education on children? Why?

        \begin{answer}
        
            No, since there is not enough variation in important characteristics
            such as years of schooling, subject studied, etc.
        
        \end{answer}
        
        \item What would be a plausible strategy to identify the causal effect
        of parent education on the outcomes of their children?

        \begin{answer}
        
            Look for a quasi-experiment by finding a policy change exogenous
            which impacts parents education and analyse whether subsets of those
            cohorts are exogenously impacted by the IV affected earnings
            outcomes for their children.
            
        \end{answer}
        
    \end{enumerate}
    
\end{enumerate}

\printbibliography
\end{refsection}

\end{document}