\documentclass[../Main.tex]{subfiles}

\begin{document}
\begin{refsection}

\chapter{Child Poverty}

\chaptertoc
\intro{
\begin{itemize} 

    \item \textit{What is the effect of living in poverty as a child?}

    \item \textit{What is the identification problem in estimating this effect?}

    \item \textit{What tools have economists used to estimate this effect?}

    \item  \textit{What is the evidence on the estimated effects of child
    poverty on outcomes?}
    
\end{itemize}

}

\section{Child Poverty in the UK}
\cont{A Big Problem}{
\begin{itemize}
    \item One in three children live in poverty in the UK. 
    \item One in two children live in poverty in lone parent or ethnic minority
    households. 
    \item Internationally, the UK has child poverty rates above the
    OECD average.
\end{itemize}
}

There are many options for policy makers to increase income in low income
families that can reduce child poverty, these usually are implemented as direct
cash transfers.



\section{Economic Theory}

Economists approach child poverty with a model that utilises a \textit{child
production function}.  

$$
Y_{cf} = h(\text{Genetics}_{cf}, \text{Time}_{cf}, \text{Money}_{cf})
$$
For a given child, $c$, born into a given family, $f$, received wage $Y$ based
on factor inputs: Genetics, time, and money). 
The function $h$ defines the process through which these inputs determine child
outcomes, usually appropriated as wage.

Time and money investments in children may change dynamically as parents income
changes, for example if the parents have an opportunity to earn more
$\text{Money}$, they may spend less $\text{Time}$ with their children. Raises
questions such as the trade off between time spent and money invested and its
allocation impacts child futures. 

However, focus is on \textbf{child poverty}, the marginal impact of money is
likely to be much greater to families that live in poverty as that unit of money
is likely to have tangible health outcomes:

    
\begin{itemize}
    \item Lowering stress in the household 
    \item Able to buy more nutritious food 
    \item By books or other supplementary materials for the child
    \item Pay off bills, alleviate heat poverty. 
\end{itemize}

First, test whether these inputs $\text{Time}$ and $\text{Money}$ increase as
household income increases. This would ensure that a policy choice of cash
transfer to households results in tangible changes to child outcomes.

There is a lot of evidence to suggest the theory is correct, increases in cash
transfers effect household consumption in ways beneficial to children.
\textcite{gregg_family_2006} and \textcite{jones_impact_2018} found that there
is no evidence that families spend cash transfers on \textit{bad} consumption
choices such as tobacco and alcohol. Furthermore, \textcite{evans_giving_2014}
showed that cash transfers in the USA improved parent physical and mental
health.



\section{Empirical Approaches}

The aforementioned studies established that cash transfers to households are
used to the child's benefit, however to further examine the policy implications
we must assess whether these cash transfers actually raise the success of
children \textit{later} in there life. If so, money matters for children in
poverty, as it impacts there life outcomes. If \textit{not}, there are likely
unobservable traits of families which cause poverty to persist
intergenerationally. We will utilise a regression equation including both parent
and child outcomes to assess this potential mechanism. 

\begin{equation}
Y_C = \alpha + \beta Y_P + \delta X + \epsilon
\end{equation}

\begin{itemize}
    \item $Y_C$ and $Y_P$ denotes wage outcomes of children and parents
    respectively
    
    \item $\beta$ represents the impact of parental wage outcomes on child wage
    outcomes
    
    \item $X$ again represents controls: including parents' education, age at
    birth, region, and intelligence. 
    
\end{itemize}

One concern in this model is the likely endogeneity of parental outcomes $Y_P$
on child outcomes which could bias the model. A glaring confounding variable
that is a potential source for this endogeneity, is \textit{region}. As a
person's income is heavily dependent on where they live, and individuals are
more likely to work where they grow up due to social ties, this would
simultaneously impact both parental and child outcomes and would create bias if
not accounted for. Region can be easily controlled for and accounted for in $X$,
if the study implements a representative dataset with households across a
country, however some variables are both difficult to measure or identify.

Some approaches used by econometricians to control for these variables include 
\begin{itemize}
    \item \textit{Quasi}-natural experiments such as local casino openings
    \autocite{akee_parents_2010} and random lottery winners
    \autocite{bleakley_shocking_2016}. 
    \item Instrument Variable Approaches 
    \item  Sibling differences 
    \item  Neighbourhood differences 
    \item  Adoptees 
\end{itemize}


\subsection{Difference in Differences}

\exm{\textcite{akee_parents_2010}}{
Akee et al. used casino opening in Eastern Cherokee in 1990s whose profits were
regularly distributed amongst all American-Indians.  These large, regular annual
transfers raised average household income in the region by 20-40\%. They used
difference in differences between two treatment groups across different time
horizons: 

\begin{description}
    \item[Control Group] were households with no American-Indian parents and
    therefore no regular cash transfers.
    
    \item[Treatment Groups] were household with either one or two A-I parents
    that received one or two payments respectively
    
    \item[Time Horizons] compared the difference in receiving the cash transfer
    for 6 years compared to none, and 2 years compared to none on the child
    outcomes. 
    
\end{description}

}

\section{Practical Questions}

\begin{itemize}

    \item[10\%] Explain why a correlation between the income of parents and their
    children (when they become adults) will not identify a causal parameter, giving
    examples of possible confounders.
    
    \begin{answer}
        There are many variables that are likely to impact both the parents and
        child's wage outcomes simultaneously. For example, region. Individuals
        are likely to work near where they grow up due to social ties and a greater
        sense of belonging in the place they grew up in, however, this also greatly
        impacts one's wage. Consequently, if not included in a regression between
        parental and child income, the OLS estimator could be biased as it has conflated
        region specific dynamics.
    \end{answer}
    
    \item [30\%] \textcite{bleakley_shocking_2016} estimate the effect of parents lottery
    win on child outcomes.
    
    \begin{itemize}
    
        \item Read the paper (focus on the introduction and read more into the
        remaining sections if it will help your understanding).
        
        \item Write the estimation equation and define the treatment and control
        groups.
    
        \begin{answer}
            The estimation equation used was, 
        
            $$Y_{ij} = \gamma T_j + \beta \mathbf{X}_{ij} + \delta_{ai} + \epsilon_{ij}$$
            
            Where $i$ is each individual, and $j$ indexes the lottery-eligible
            person. $T_j$ represents the treatment, where a person $j$ receives
            a winning parcel, $\mathbf{X}$ represents a set of controls,
            $\delta_{ai}$ are a set of age dummies. The treatment group are the
            set of individual's who won the lottery $T = 1$ compared to the
            control group who did not receive a lottery parcel $T=0$.
            
        \end{answer}
        
        \item What were the findings of the paper (read through the results
        section to understand how the authors discuss the results. Be sure to
        quote and explain the estimated coefficients.
    
        \begin{answer}
        
            Overall, they found contrasting evidence to the theory: ``Children
            of lottery winners did not have more wealth, literacy, or income as
            adults." The paper's discussion focuses on two key questions, the
            comparative returns to skill for lottery winner's children and
            intergenerational wealth of lottery winners. They find for the
            former that families did not invest more in their children's human
            capital after winning a lottery (their estimates for literacy rates
            amongst the treatment group were insignificant). For the latter,
            they find that ``the elasticity of son's 1870 wealth with respect to
            the father's 1850 wealth is between 0.13 and 0.28", implying that a
            1\% increase in father's wealth in 1850 corresponded to between a
            0.13\% - 0.25\% increase in the son's wealth, although statistically
            significant, this small amount implies the intergenerational wealth
            mechanism is weak and that individual factors have a larger
            explanatory role in determining a child's wage outcomes. 
            
        \end{answer}
        
        \item Do the findings of this paper generalise to the question: does
        living in poverty as a child affect long-run outcomes for the child? (Be
        creative with your answers – I am interested in your thoughts).
    
        \begin{answer}
        
            There is considerable doubt that \textcite{bleakley_shocking_2016}
            findings generalise to whether poverty as a child affect long-run
            outcomes for children. Firstly, the study's dataset and analysis is
            tied to a small temporal-spatial context, that is, a county in
            Georgia during the 18$^{\text{th}}$ century over a period of
            significant economic turbulence. The author's note how the Civil War
            and Slavery Emancipation altered capital accumulation and investment
            allocation which could distort the behaviour of parents who received
            lottery payments. Therefore, it is unlikely that these findings
            could be generalised to a modern context or in other regions which
            have different social dynamics. For example, the increase in
            education over the last century and the declining birth rates is
            likely to have altered familial dynamics; parents may place greater
            weight on child outcomes when they have fewer children, or if they
            are more educated about the lasting impact parental support has on
            the development of a child.
    
        \end{answer}
        
    \end{itemize}

\item [30\%] \textcite{akee_parents_2010} wanted to estimate the effect of
family income on the outcomes of children once they have grown up. The authors
were concerned that simply correlating family income with child outcomes would
produce a biased estimate. Explain the methods they used to identify the causal
effect. What did their study find?

\begin{answer}
    
They used a difference in difference approach comparing the child outcomes whose
parents received exogenous wealth shocks compared to those that did not across
two different time periods (2 years of wealth shocks and 6 years of wealth
shocks). The exogenous wealth shock identified were annual payments made by
casino's in the South Cherokee region to native American adults. Since casino
openings were exogenous to parent's income who didn't work in those regions (the
author did acknowledge a potential impact on wage demand, however after
controlling from distance to the casino this was insignificant) this would
isolate for the causal impact of income changes whilst holding all other factors
constant, identifying the causal affect of family incomes on children.

Children in affected households have higher levels of education in their young
adulthood and a lower incidence of criminality for minor offenses. Effects
differ by initial household poverty status. An additional \$4000 per year for
the poorest households increases educational attainment by one year at age 21
and reduces having ever committed a minor crime by 22\% at ages 16-17. Their
evidence suggests that improved parental quality is a likely mechanism for the
change.

\end{answer}

\end{itemize}
\printbibliography
\end{refsection}
\end{document}

        