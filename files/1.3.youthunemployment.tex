\documentclass[../Main.tex]{subfiles}

\begin{document}

\chapter{Youth Unemployment}

\intro{


}

\section{Unemployment Scarring}

When an individual is unemployed we expect a reduction in their wage temporary, however is there longer lasting harm - like an unemployment scar? 

In microeconomics we refer to state dependence, where individual previous
``states" like education, employed, unemployed, or out of the labour force drive
different aspects of unemployment. 

For the unemployed state, scholar argue that there is \textit{state persistence}
where individuals who become unemployed remain unemployed.

{Importance of Youth Unemployment Scar} 

``Solving youth unemployment is those pressing problem governments are facing
today. Not dealing with the problem of high, and rising levels of youth
unemployment hurts the youngsters themselves and has potentially severe
consequences for us all for many years to come. The time to act is now. The
young must be the priority." - Bell and Blanch-flower (2010). 



\end{document}